\documentclass[12pt, a4paper]{report}

\usepackage[
left=2.5cm, 
right=2.5cm, 
top=2.5cm, 
bottom=2cm, 
%headsep=\dimexpr (3.5cm-50pt)/2 -10pt \relax, 
%headheight=50pt,
]{geometry}

% modern latin font
\usepackage{lmodern}

% improved typography
\usepackage{microtype}

% set text spacing
\usepackage[onehalfspacing]{setspace}

%german language support
\usepackage[ngerman]{babel}

% customize indentation
\setlength{\parindent}{0pt}
\setlength{\parskip}{0.3em}

% customize section headers
\usepackage[compact]{titlesec}
\titleformat*{\subsection}{\bfseries}
\titlespacing*{\subsection}{0pt}{5pt}{3pt}
\titleformat{\chapter}{\normalfont\huge}{\textbf\thechapter}{20pt}{\huge\textbf}
\titlespacing*{\chapter}{0pt}{-20pt}{20pt}
%\titlespacing*{\tableofcontents}{0}{-20pt}{20pt}



% table of contents spacing
\usepackage{tocloft}
\setlength{\cftbeforesecskip}{0.6cm}
%\setlength{\cftparskip}{0pt}
%\setlength{\cftbeforesubsecskip}{0.2cm}
\renewcommand{\cftaftertoctitle}{%
    \\%[0.2cm]
    \mbox{}\hfill{\bfseries Seite}}
\setlength\cftbeforetoctitleskip{0pt}
\setlength\cftaftertoctitleskip{-4pt}

% includegraphics
\usepackage[]{graphicx}
\usepackage{float}

% captions
\usepackage[labelfont=bf]{caption}

% tablular
\usepackage{tabularx}
\usepackage{multirow}
\usepackage{makecell}

% customize equations
\usepackage[fleqn]{amsmath}
\numberwithin{equation}{section}

% customize math intertext
\usepackage{mathtools}
\mathtoolsset{
   above-shortintertext-sep=2ex,
   below-shortintertext-sep=2ex,
}

% si units
\usepackage{siunitx}

% enumerations
\usepackage{enumitem}
\setlist{left = 0pt, font = \bfseries}

%color support
\usepackage[dvipsnames]{xcolor}

%code environments
\usepackage{listings, multicol}
\usepackage{tcolorbox}
\tcbuselibrary{listings, breakable, skins}
\usetikzlibrary{shadings}
\tcbset{
   listing engine = listings,
}

\usepackage{placeins}
% customize links

% more whitespace in tables
\setcellgapes{5pt}
\makegapedcells

% new font
\usepackage{helvet}

\usepackage{pgfplots}
\pgfplotsset{compat=1.17}

\usepackage[european,straightvoltages]{circuitikz}
\usetikzlibrary{calc}
\usetikzlibrary{patterns}

\usepackage{hhline}

% schöne hochgestellte Quellenverweise
%\usepackage[superscript]{cite}
\bibliographystyle{unsrtdin}
\usepackage{mathtools}


% set pagestyle
\usepackage{fancybox,fancyhdr}
\pagestyle{plain}
\lhead{\textsc{Lfr}}
\rhead{Raddatz, Eismann, Buck, Hecker}
\cfoot{\thepage}
\setlength{\headheight}{15pt}


\fancypagestyle{fancyempty}{
    \fancyhf{}
    \renewcommand{\headrulewidth}{0pt}%
    \renewcommand{\footrulewidth}{0pt}%
}

\setcounter{secnumdepth}{5}
\setcounter{tocdepth}{5}

\usepackage[hyphens]{url}

\definecolor{blue1}{RGB}{100,149,237}
\usepackage{hyperref}
\hypersetup{
   allbordercolors=blue1,
   colorlinks=false,
   pdfborderstyle={/S/U/W 1},
}
%Bild auf Deckblatt

\title{Line Following Robot - Dokumentation}
\author{Fynn Raddatz, Erik Eismann, Julian Buck, Yannick Hecker}


\renewcommand{\arraystretch}{1.5}
\setlength{\tabcolsep}{20pt}
%\definecolor{keywordcolor}{RGB}{0,0,153}
\definecolor{numbers}{gray}{0.1}
\definecolor{vlgray}{gray}{0.9}
\definecolor{lgray}{gray}{0.2}
%\newcounter{definitioncounter}


\lstset{
	basicstyle=\ttfamily,
	keywordstyle={\color{keywordcolor}\bfseries},
	ndkeywordstyle={\color{BurntOrange}},
	stringstyle={\color{ForestGreen}\ttfamily\textit},
	identifierstyle=\color{black},
	commentstyle={\color{gray}\ttfamily},
	emphstyle={\color{OrangeRed}},
%	numbers=left,
%	numberstyle=\tiny\bfseries\color{numbers},
	showstringspaces = false,	
%	stepnumber = 2,
	breaklines,
}

\lstdefinelanguage{Kotlin}{
	comment=[l]{//},
	emph={filter, first, firstOrNull, forEach, lazy, map, mapNotNull, println},
	keywords={!in, !is, abstract, actual, annotation, as, as?, break, by, catch, class, companion, const, constructor, continue, crossinline, data, delegate, do, dynamic, else, enum, expect, external, false, field, file, final, finally, for, fun, get, if, import, in, infix, init, inline, inner, interface, internal, is, lateinit, noinline, null, object, open, operator, out, override, package, param, private, property, protected, public, receiveris, reified, return, return@, sealed, set, setparam, super, suspend, tailrec, this, throw, true, try, typealias, typeof, val, var, vararg, when, where, while},
	morecomment=[s]{/*}{*/},
	morestring=[b]",
	morestring=[s]{"""*}{*"""},
	ndkeywords={@Deprecated, @JvmField, @JvmName, @JvmOverloads, @JvmStatic, @JvmSynthetic, Array, Byte, Double, Float, Int, Integer, Iterable, Long, Runnable, Short, String, Any, Unit, Nothing},
	sensitive=true,
}


\newtcbinputlisting{\Kotlin}[4][]{%
%	step=definitioncounter,
%	frame style={upper left=vlgray, lower right=lgray},
	left=6mm,
	listing file={code/Kotlin/#2},
	colback = vlgray,
	colframe = vlgray,
	enlarge top by=0mm, top=-2mm, bottom=2mm, 
	enhanced,
	after={\par\vspace{0.5\baselineskip}\noindent},
%	overlay={\node[
%		draw,
%		fill= lgray,
%		yshift=4pt,
%		xshift=-25pt,
%		left,
%		text=white,
%		anchor=east,
%		font=\footnotesize\bfseries,
%%		rounded corners,
%		] at (frame.south east)
%		{#2};
%	},
	listing only,
	listing options={
		language = Kotlin,
		columns = fullflexible,
		firstnumber = {#3},
		firstline ={#3},
		lastline = {#4},
	},
	breakable,
%	sharp corners = southeast,
	#1
}

\newtcbinputlisting{\CPP}[4][]{%
%	width=0.8\textwidth,
%	step=definitioncounter,
	left=6mm,
	frame style={upper left=vlgray, lower right=lgray},
	center,
	listing file={code/C++/#2},
	colback = vlgray,
	colframe = vlgray,
	enlarge top by=0mm, top=-2mm, bottom=2mm, 
	enhanced,
	listing only,
	after={\par\vspace{0.5\baselineskip}\noindent},
	overlay={\node[
		draw,
		fill= lgray,
		yshift=4pt,
		xshift=-25pt,
		left,
		text=white,
		anchor=east,
		font=\footnotesize\bfseries,
		%		rounded corners,
		] at (frame.south east)
		{#2};
	},
	listing options={
		language = C++,
		columns = fullflexible,
		firstnumber = {#3},
		firstline ={#3},
		lastline = {#4},
%		multicols=2,
	},
	breakable,
%	sharp corners = southeast,
	#1
}

 \renewcommand{\familydefault}{\sfdefault}
\begin{document}

    \begin{figure}
        \centering
        %\includegraphics[width=0.9\linewidth]{img/strasse1}
    \end{figure}
    
   \vspace*{1.2cm}
   \begin{center}
       {\huge SiWaSim - Handbuch}\\[0.8cm]
       \large{Automatisierte Tests für SIWAREX-Module}\\[0.5cm]
	
   \end{center}
   \vfill
   \begin{flushleft}
       \large{Autor: Fynn Raddatz}\\
       \today
       \thispagestyle{empty}
   \end{flushleft}
   \newpage
   \tableofcontents
   %
   \clearpage
   %
   \pagestyle{fancy}
   \fancypagestyle{plain}{%
       \lhead{\textsc{SIWASIM}}
       \rhead{\empty}
       \cfoot{\empty}
       \rfoot{\thepage}
       \lfoot{Fynn Raddatz}
       \renewcommand{\headrulewidth}{1pt}%
       \renewcommand{\footrulewidth}{1pt}%
   }

\chapter{Einleitung}
\section{Begleitende Dokumente}
Begleitend zu diesem Handbuch stehen weitere Dokumente zur Verfügung. Diese befinden sich standardmäßig in der Ordnerstruktur des vollständigen Projekts. Alle Dokumente und ihr Inhalt sind im Folgenden aufgeführt.
\begin{itemize}
\item SiWaSim - Handbuch: Handbuch mit Features, Bedienung- und Sicherheitshinweisen für den Endbenutzer
\item Design Referenz: Dokument mit Beschreibung des Aufbaus und der Funktionsweise des Sourcecodes und der Hardware, zur Einarbeitung bei Weiterentwicklung des Simulators
\item Doxygen: Durch Doxygen generierte HTML- und \LaTeX -Dokumente zur Dokumentation der im Code enthaltenen Klassen und Funktionen
\item Feature-Spec: Dokument mit Zielsetzungen und Funktionsumfang des Simulators vor Entwicklungsbeginn
\item IA-Board: Ordner mit Dokumentationen zum verwendeten IA-Board der Firma Sequent Microsystems vom Hersteller
\item PCB: Enthält die Schaltpläne und das EDA-Projekt für das angefertigte PCB
\item Bilder: Enthält Bilder vom Simulator und der Platine
\item Alte Referenzen: Enthält alte Dokumente die teilweise als Vorlage oder als Einarbeitung verwendet wurden
\end{itemize}
\section{Beschreibung}
Bei dem Produkt „SiWaSim“ (SIWAREX Simulator) handelt es sich um ein Gerät auf RaspberryPi-Basis, welches die Eigenschaften einer Wägezelle und weitere anlagenspezifische Eigenschaften simuliert. Hiermit kann ein Aufbau mit einem SIWAREX-Modul, optional in Kombination mit anderen Komponenten (SPS, HMI, etc.), im Laborumfeld auf die richtige Funktionsweise automatisiert geprüft werden. Dazu gehören u.a. Gewichtssimulationen, Steuern von Digitalausgängen, Bandgeschwindigkeitssimulation und die damit verbundenen Überwachungen, Überprüfung auf Plausibilität und das Logging der Messwerte.
\section{Aufbau}
Der Simulator besteht aus einem RaspberryPi 4B 4GB. Auf diesem läuft die gesamte Simulationssoftware. Er ist für die Ansteuerung aller Hardwarekomponenten sowie für die optionale Kommunikation nach Außen mit einem PC zuständig. Auf dem RaspberryPi befindet sich ein Industrial Automation Board (IA-Board) der Firma Sequent Microsystems, welches über einen I2C-Bus angesteuert wird. Das Board verfügt über diverse digitale sowie analoge Ein- und Ausgänge. Zusätzlich lässt sich hierüber das RS485 Interface des SIWAREX-Moduls ansteuern, wodurch alle verfügbaren MODBUS-Kommandos verwendet werden können. Als dritte Hauptkomponente wurde ein eigenes PCB angefertigt, das das IA-Board mit den Anforderungen der SIWAREX-Module kompatibel macht. Außerdem wird auf diesem PCB die Simulation der Wägezelle hardwareseitig umgesetzt.
\chapter{Funktionsumfang}
\section{Wägezellen-Simulation}
Der Simulator verfügt über einen DSUB-9 Stecker auf der obersten Platine. Dieser ist kompatibel mit dem im Labor standardmäßig genutzten Adapterkabel, welches an die Klemmen des SIWAREX-Moduls angeschlossen werden kann. Die folgenden Eigenschaften einer Wägezelle können simuliert werden.
\subsection{Wägezellen-Charakteristik}
Es können Wägezellen von $1\mathrm{mV/V}$ bis $4\mathrm{mV/V}$ simuliert werden. Hierzu gehören auch Wägezellen von $-1\mathrm{mV/V}$ bis $-4\mathrm{mV/V}$.
\subsection{Wägezellen-Impedanz}
Es lässt sich die Impedanz der Wägezelle, also die Impedanz zwischen den Leitungen EXC+ und EXC- simulieren. Hierbei sind drei Zustände möglich:
\begin{itemize}
\item Open-Circuit, zum Beispiel durch Kabelbruch
\item Nominal, standardmäßige Impedanz von etwa 350$\Omega$
\item Short-Circuit, Kurzschluss der EXC-Spannung
\end{itemize}
\subsection{Überlast}
Durch Hardwarejumper kann der maximale Bereich der Ausgangsspannung leicht erhöht werden. Somit kann ein Übergewicht von etwa 10\% simuliert werden.
\subsection{EXC-Spannung}
Der Simulator ist kompatibel mit EXC-Spannungen bis 10V. Eine geringere Spannung verringert die Auflösung der simulierten Zellspannung. Außerdem wird die Wägezellenspannung verringert, wenn die EXC-Spannung sinkt. 
\subsection{SEN-Spannung}
Standardmäßig sind SEN+ und EXC+ auf dem PCB verbunden, sodass ein Sechsleiterbetrieb möglich ist. Hier wird die Sense-Spannung, wie in einer normalen Anlage, an der Impedanz der Zelle abgegriffen. Es kann aber auch eine Sense-Spannung, unabhängig von der EXC-Spannung, simuliert werden. Hier kann eine Spannung zwischen 0V und 10V ausgegeben werden.
\chapter{Benötigte Software}
Zur Bedienung des Simulators ist die Installation einiger Anwendungen nötig. Im Folgenden werden Anwendungen genannt, dessen korrekte Funktion mit dem Simulator durch eigene Nutzung überprüft wurde. Andere alternative Lösungen sind natürlich auch möglich, sind aber ungetestet. Benötigt wird:
\begin{itemize}
\item PuTTY: Zur SSH Verbindung mit dem RaspberryPi zur Verwendung der Konsole
\item WinSCP: Zum Laden von Dateien auf den RaspberryPi via FTP
\item SiWaSim PC-Software: Zur komfortablen Konfiguration und Überwachung
\item Excel: Zur Konfiguration, sofern die eigene Windows-Software nicht verwendet wird
\item Ein Texteditor zum Erstellen von Python-Skripten für automatisierte Tests
\end{itemize}
\chapter{Inbetriebnahme ohne SiWaSim PC-Software}
Im folgenden Kapitel wird erläutert, wie der Simulator eingeschaltet und konfiguriert werden kann, sodass erste Grundfunktionen nutzbar werden. Dieses Kapitel bezieht sich auf die Inbetriebnahme ohne die eigene SiWaSim PC-Software. Es wird empfohlen, diese Software zur komfortablen Konfiguration zu nutzen.
\section{Vor dem Start}
\subsection{Hardwarekonfiguration}
Bevor der Simulator in Betrieb genommen wird, müssen einige Hardwarejumper basierend auf der gewünschten Simulation gesetzt werden. 
Auf der obersten Platine befinden sich vier dieser Jumper. Diese sind mit V1, V2, V3 sowie SEN beschriftet. Die Jumper müssen basierend auf den folgenden Fragen gesetzt werden.\\\\
Soll (1.) die Sense-Spannung (SEN) direkt mit der Versorgungsspannung EXC verbunden werden, oder soll (2.) die Sense-Spannung unabhängig von EXC über einen Analogausgang eingestellt werden?
\begin{itemize}
\item[(1.)] SEN Jumper auf EXC (untere zwei Pins) 
\item[(2.)] SEN Jumper auf SIM (obere zwei Pins) 
\end{itemize}
Soll eine negative Zellcharakteristik (z.B. -4mV/V) simuliert werden?
\begin{itemize}
\item[Ja:] V1 und V2 nach rechts setzen (jeweils die beiden rechten Pins verbinden)
\item[Nein:] V1 und V2 nach links setzen (jeweils die beiden linken Pins verbinden)
\end{itemize}
Soll die Gleichtaktspannung der SEN-Leitungen auf (1.) EXC/2 gesetzt werden oder soll (2.) die Gleichtaktspannung manuell über einen Analogausgang gesetzt werden?

\begin{itemize}
\item[(1.)] V3 nach links setzen (jeweils die beiden linken Pins verbinden)
\item[(2.)] V3 nach rechts setzen (jeweils die beiden rechten Pins verbinden) 
\end{itemize}
Die dritte Frage ist nur relevant, wenn bei der zweiten Frage NEIN gewählt wurde.
\subsection{Verdrahtung}
Folgende Verdrahtungen sind vorzunehmen:
\begin{itemize}
\item Versorgungsspannung (24V) anschließen (unten rechts auf dem PCB)
\item Wägezellenkabel an den DSUB-9 Stecker anschließen und mit SIWAREX-Modul verbinden
\item Digitalausgänge des SIWAREX-Moduls an Digitaleingänge der Platine anschließen 
\item Optional: Digitaleingänge des SIWAREX-Moduls an Digitalausgänge der Platine anschließen 
\item Optional: Für Funktionen wie WebServer oder WriteProtect SW1 und SW2 verbinden (unten rechts)
\item Optional: Für schaltbare Stromversorgung (z.B. für Power-Cycle) POWER SW verbinden
\item Optional: Für Bandwaage Geschwindigkeitsausgang (PWM Out) an CNT-Eingang des SIWAREX-Moduls
\item Optional: Analog-Strom-Ausgang an Analog-Strom-Eingang des SIWAREX-Moduls (Klemmen auf zweiter Ebene) 
\item Optional: Analog-Strom-Eingang an Analog-Strom-Ausgang des SIWAREX-Moduls (Klemmen auf zweiter Ebene)
\item Optional: Für RS485 MODBUS Signalleitungen verbinden (Klemmen auf zweiter Ebene) 
\end{itemize}
\section{Erstellen und Laden von Konfigurationen}
\section{Manuelle Steuerung}
\chapter{Automatisierte Tests}
\section{Einleitung}
\section{Erzeugen von Abläufen}
\chapter{Technische Daten}

\end{document}
